\chapter{Introduction}
\label{chapter:intro}
% The target
% ==========
This thesis handles the problem of clustering Finnish scientific 
publications by their metadata. The target is to cluster 
them as well as possible by their scientific discipline. \fixme{
Tahan enemman tavoitteen kuvausta ei viela maarittelyn vaikeutta.}
Of course 
the ``wellness'' of a clustering measured by how it fits to 
scientific fields is a tricky issue for at least couple of 
reasons.
% Testirivi merkkikoodauksen kokeilua varten varten



% Obstacles
% =========
First, there is no general consensus about what is the correct 
partition of all science to different branches. It might be depend 
on specific need or individual opinion where the line between two 
related discipline lies.

Second, science is evolving all the time. What was yesterday seen 
simply as chemistry could today be viewed as organic and inorganic 
chemistry.

Third, the definition of scientific disciplines depends also how 
closely we look into a topic. Sometimes chemistry is sufficent 
description for eg. a publication and sometimes we need to define 
it more accurately as organic chemistry.
% LMa: Tahan maininta poikkitieteellisten alojen 
% ongelmallisuudesta, onko bioinformatiikka, biologiaa, 
% tietotekniikkaa vai oma tieteenalansa.

But despite this ambiguousness of the target we define and 
justify some goals and how to measure our success.


% Environment/background
% ======================
In Finland there are 15 universities and 23(+2) universities of 
applied sciences. Additionally there are 12 research institutes.
Each year Finnish scientific efforts produce about 
10000 publications in all scientific disciplines. 
\fixme{10000 tulee siis WOS ja Elsevierilta} The Ministry of 
Education and Culture oversees what is researched and in what 
quanities in the Finnish research.
Classifying articles by discipline enables different types of 
bibliometric analysis.
% It is also ill posed problem because there is no one right answer
% for the problem. Different tasks require different classification.
Because of the amount of the scientific articles manual 
classification is not applicable. \fixme{Also: kukaan tuskin 
pystyy hallitsemaan kaikkia tieteenaloja. Tavoitteena tyokalu, 
jolla analyysin tekija voi luokitella }
% All these are manual methods. Automatic methods are needed...
Automatic methods should be able to label the article by some 
criterion to the (subjectively) obvious discipline. The input for 
automatic method can usually at most be the whole article and 
perhaps some metadata describing it. The metadata can be 
created by the author or the publisher or some archive.

\fixme{What methods used?}
Different methods have been suggested. NN1 suggested SOME METHOD. 
NN2 suggested SOME OTHER METHOD. NN3 suggested YET A METHOD.

In this thesis we will try to cluster Finnish publications by 
discipline by using the metadata.





\section{Problem statement}

Refactor from above...

\section{Structure of the Thesis}
\label{section:structure} 

% Use transition in your text, meaning that you should help
% the reader follow the thesis outline. Here, you tell what will be in
% each chapter of your thesis.
In chapter 2. we will shortly present the background...
