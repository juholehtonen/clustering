\chapter{Introduction}
\label{chapter:intro}
% enemman tavoitteen kuvausta ei viela maarittelyn vaikeutta
% The goal
% ==========
This thesis handles the problem of clustering Finnish scientific 
publications by their metadata. The goal is to cluster 
them as well as possible by their scientific discipline. We want 
to find a clustering for the publications such that it matches 
as well as possible an imaginary average manual classification 
done by human experts.

% Why this goal is needed?
% ========================
Classifying publications (such as research articles) by discipline 
enables different types of bibliometric analyses. Such 
classifications can be used in summarizing research output of a 
university or a country. The sum of all scientific publication of 
an individual country for example can be interesting and useful 
information that can tell something about research and its level 
in that country.
% Eri tutkimusalojen ja niiden julkaisumäärien tarkastelu esimerkiksi 
% kansallisten koulutuspäätösten tueksi tarvitsee bibliometriikka.
%In Finland the ministry of education and culture uses such 
%information to follow what is researched and in what quanities in
%the Finnish academia.

We use Finnish scientific publications from  Web of Science (WoS)
database as our source data. 
Web of Science database gets approximately 10000 publications 
annually that can be assigned as a result of Finnish research, that
is at least one author is affiliated to a Finnish research 
organisation (see \cite{auranen_tieteen_2018} for definition).
% https://www.aka.fi/globalassets/42julkaisut/aka_tieteen_tila_2018_web.pdf
% Liite 1 Bibliometrinen aineisto ja menetelmät
It should be noted that not all Finnish research can be found from
WoS database. Especially humanities and social sciences are 
underrepresented so it isn't comprehensive image of Finnish 
research.
% It is also ill posed problem because there is no one right answer
% for the problem. Different tasks require different classification.
Due to the amount of the scientific publications, manual 
classification is not applicable. It is also impossible for any 
one person to master all fields of science to manage such a
task. Our goal is a method that the analyst herself can apply.
% Environment/background/context
% ==============================

% In Finland there are 13 universities, 12 state research institutes,
% and 24 universities of applied sciences. Additionally there are 
% other instances like military academy and private companies that 
% contribute to research. 
For clustering the publications we will use metadata that the authors 
of the publication will provide in any case with their work such 
as the title, the abstract and the keywords. 
% Ryvästykseen käytämme Wardin hierarkkista klusterointia.
Clustering will not produce any meaningful labels for the 
clusters. They must be labeled appropriately by other means. \fixme{We 
will label the produced clusters using the top five most 
significant terms of a cluster.}

The ``wellness'' of a clustering measured by how it fits to 
scientific fields is a tricky issue for at least couple of 
reasons.
% Obstacles
% =========
First, there is no general consensus about what is the correct 
partition of all science to its different branches, or even what
are the different fields of science. It might depend 
on specific need or individual opinion where the line between two 
related discipline lies.
Second, science is evolving all the time. What was perhaps at some 
point in history seen simply as chemistry is today commonly 
divided at least to organic and inorganic chemistry.
Third, the definition of scientific disciplines depends also how 
closely we look into a topic. Sometimes chemistry is sufficent a
classification for example for a publication and sometimes we need 
to label it more accurately as organic chemistry.
Fourth, there is a considerable amount of research that is 
intrinsically cross-disciplinary. Bioinformatics for example relies
heavily on biology and computer science at least. So it depends on the 
purpose whether a publication should be considered belonging to
biology, computer science, bioinformatics or to all of them.

\fixme{We will measure the resulting clustering by comparing it to the
existing Web of Science classification. The clustering is good if
it in outline matches WoS classification.}
% ...mutta kenties tarkentaa sitä ryvästämällä jotkut julkaisut 
% pienempiin ryppäisiin samankaltaisten julkaisujen kanssa.
% But despite this ambiguousness of the target we define and 
% justify some goals and how to measure our success.


% All these are manual methods. Automatic methods are needed...
% Automatic methods should be able to label the article by some 
% criterion to the (subjectively) obvious discipline. The input for 
% automatic method can usually at most be the whole article and 
% perhaps some metadata describing it. The metadata can be 
% created by the author or the publisher or some archive.

% \fixme{What methods used?}
% Different methods have been suggested. NN1 suggested SOME METHOD. 
% NN2 suggested SOME OTHER METHOD. NN3 suggested YET A METHOD.

% \section{Structure of the Thesis}
% \label{section:structure} 

% Use transition in your text, meaning that you should help
% the reader follow the thesis outline. Here, you tell what will be in
% each chapter of your thesis.
In Chapter 2. we will shortly present the general concepts of 
bibliometrics and some efforts in the field in Finland. We will 
also present the main focus of this thesis, clustering.
Next in Chapter 3. we describe the data and present the methods we 
will use to analyze it.
In Chapter 4. we go through the steps of the analysis in practice 
describing all the choices we made.
After that in Chapter 5. we present the results and finally 
summarize all in Chapter 6.











