\chapter{Implementation}
\label{chapter:implementation}
Here we describe how the methods are implemented to achieve the 
clustering. First we describe how the raw text data is 
pre-processed 

We implemented a workflow for clustering using Python's 
scikit-learn package. \fixme{Include a graph visualizing the 
workflow.}

First data was read from raw files obtained from publishers. 
%(Actually YL probably also preprocessed data before passing it 
% to me.)
Unrelevant metadata fields were omitted and the relevant were read 
into a dictionary. We choose to keep title, abstract and keyword
fields. Terms in keyword fields were concatenated to combined 
terms: ``allergic\_contact\_dermatitis''. Another option would be to
count n-grams to preserve the meaning of combined terms. (Justify 
decision.) Title and abstract fields were treated with a bag of 
words representation. (Explain.)



Data was lemmatized. We choose to remove all plain numbers from
data. Another option is to replace numbers with dummy '\#NUMBER'.
That could help separate natural sciences from humanities. Because 
we didn't have proper knowledge on the issue removal was chosen. So 
numbers would be ignored. That would be a more neutral way to treat 
them. (How? Explain...)

Running preliminary tests for 2000 publications with tf-idf's maximum and 
minimum document frequencies set to $max_df=0.5$ 
and $min_df=2$ (integers denote absolute count value) respectively
resulted a feature space of 3372 terms and a stop word list of 7475 
terms. The stop word list included 352 a priori set words and words
that occurred in too many ($max_df$) or too few ($min_df$) 
documents. Terms that were filtered out as too frequent or too few
were for example: ``haemoglobin\_a'', ``jacobian\_matrix'', 
``parthenogenetic'', ``chelation\_by\_saccharide\_molecules'', 
``leg\_blood\_supply'', ``computational\_fluid\_dynamics'', 
``pigment'', ``hdtv'', ``nicotinic\_receptor''.

(Interesting would be to know the frequencies of the filtered-out 
terms. It didn't come out straight from TfIdfVectroizer class used 
but taking different levels of $min_df$ and $max_df$ could be used 
to view different frequency groups.)

When trying with $max_df=0.6$ we get feature space of XXXX and 
YYYY stop words.

