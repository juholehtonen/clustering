\chapter{Background}
\label{chapter:background}

In this chapter we briefly introduce clustering and how it is 
positioned in the larger field of machine learning. But first we
describe what bibliometric is.

\section{Bibliometric}
\label{sec:bibliometric}

Bibliometric is a study of written scientific records... 
\fixme{citation}

\fixme{Nykyiset bibliometriikan kayttamat menetelmat 
kehittyneempia yhdistetty verkkoanalyysia ja klusterointia.}



\section{Clustering}
\fixme{Alkuun koneoppimisesta, ohjaamatonta ja ohjattua. Eri 
ohjaamattomat menetelmat. Mihin kaikille aloille klusterointi voi 
kayttaa.}

\fixme{Ero verkkoanalyysiin: lyhyesti}


\subsection{Choosing the number of clusters}
For K-means the choosing the number of clusters must be done before
the clustering. This is an unsolved problem. \fixme{citation}
Usually the 
problem is dealt by running the algorithm with various number of 
clusters and then measuring each by some metric. \fixme{citation}

In hierarchical clustering number of clusters can be chosen after 
the data has been clustered and the dendogram formed. 
\fixme{citation} By 
looking the dendogram and using different metrics the number of 
clustrs can be chosen.


\subsection{Evaluating of clustering results}
Calinski \& Harabazt criterion can be used as an evaluation 
criterion for deciding the number clusters \fixme{citation}. 
