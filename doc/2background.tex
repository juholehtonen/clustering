\chapter{Background}
\label{chapter:background}

% Tiedonhankintaa suunnitellessasi voi miettiä vastauksia mm. 
% seuraaviin aiheen määrittelyä selventäviin tutkimuskysymyksiin:
% 
%     Mistä aiheesta tietoa tarvitaan?
%       bibliometriikasta, sen määritelmästä sekä klusteroinnista
%     Mihin tarkoitukseen tietoa tarvitaan?
%       Aiheen taustan kuvailuun, menetelmien kuvailuun ja 
%       valitun menetelmän toteuttamiseen
%     Mikä aiheessa on keskeistä?
%       Klusterointimenetelmän kokeilu ja tulosten raportointi
%     Mistä näkökulmasta aihetta lähestytään?
%       Käytännön implementaation kokeilulla
%     Mitä aiheesta tiedetään jo ennalta?
%       Klusteroinnista perusteet, bibliometriikasta vähemmän
%     Tarvitaanko yleis- vai tieteellistä tietoa?
%       Bibliometriikasta tarvitaan vähän yleistietoa, 
%       klusteroinnista tieteellistä.
%     Tarvitaanko kuva-aineistoa?
%       Ei muuta kuin itse tuotetut
%     Minkä ikäistä tietoa tarvitaan?
%       Yleis- ja taustatiedot vanhoista asioista, aiheen 
%       oleellinen tieto uusinta
%     Minkä kielistä tietoa tarvitaan?
%       suomi ja englanti käy


In this chapter we briefly introduce clustering and how it is 
positioned in the larger field of machine learning. But first we
describe what bibliometrics is.

\section{Bibliometrics}
\label{sec:bibliometrics}

Bibliometrics is a study of written scientific records. The 
records may be books, articles, letters in scientific journals, 
conference papers and so on. Bibliometrics studies how these 
products of research are communicated, how are they related to 
each other, what kind of properties they have and what can be 
learned about the science in general by analysing these written 
scientific artifacts.

% Tama on kompelosti ilmaistu, korjaa
These reserch products and relations between them form a 
network that can reveal something about the structure of 
different scientific disciplines. Finding the structure of this 
kind of data set is called classification or clustering.

\fixme{Nykyiset bibliometriikan kayttamat menetelmat 
kehittyneempia yhdistetty verkkoanalyysia ja klusterointia.}

\subsection{Classification in bibliometrics}
% Specific charasteristics of classifying the bibliometrics
Classification in bibliometrics can be used to research the 
different fields of science and their development.
% Background/history
% ==================

\fixme{When bibliometric started?}
There has been lot of study in automatically classificating the 
science. 
%Classifying things is the first thing to do when trying to 
%understand the world. \cite{} 
%First classifications were made when humans wanted to name 
%entities of the world.
% were made by romans/greeks/sumerians... 
% The whole field of science was comprehended differently then.
The need for some kind of bibliometric indices rise in the 
First modern(?) classification was... by... some 
indexing/publishing/to facilitate communication...

% Existing classification systems
% ===============================
Currently the most popular classification system is the Thomson 
Reuter's Web of Science classification. This classification 
system classifies the journals into one or more research areas. 
\cite{waltman_new_2012} 
% It is also manually curated by the publisher. \fixme{Is it? How 
% does it work?}
Also an independent journal level classification system has been 
developed. \cite{archambault_towards_2011}
Also some discipline specific classification systems exists such 
as (check them...).

% Methods used in bibliometrics
% ===============================
Previously mentioned classification systems are often used as a 
basis and reference in bibliometric research. When 
researching new classification systems bibliometric research 
uses mainly three types of methods; citation based, text analysis 
based or combination of the two \cite{janssens_hybrid_2009}.
% Tama viittaus .bib:ssa hajoittaa koko roskan. Korjaa ensin 
% tiedostojen enkoodaukset ->UTF8 + maarita thesis.tex:ssa
Citation based methods study citations of publications and 
networks that forms when publications are connected by a 
citation. Connection between two publications can be 
formed by a direct citation, bibliometric coupling where 
publications are connected when a thrid publication cites them 
both or co-citations where poblications are connected if they 
cite the same third publication.
Text analysis based methods examine the title, the abstract or 
the whole text content of the publication itself and classifies 
the publications or journals by the topic model created 
\cite{blei_latent_2003}.
Hybrid models combine both approaches.

% Situation in Finland

\section{Clustering}
\fixme{Alkuun koneoppimisesta, ohjaamatonta ja ohjattua. Eri 
ohjaamattomat menetelmat. Milla kaikille aloilla klusterointia 
voi kayttaa.}

\fixme{Ero verkkoanalyysiin: lyhyesti}


\subsection{Choosing the number of clusters}
For K-means the choosing the number of clusters must be done before
the clustering. This is an unsolved problem. \fixme{citation}
Usually the 
problem is dealt by running the algorithm with various number of 
clusters and then measuring each by some metric. \fixme{citation}

In hierarchical clustering number of clusters can be chosen after 
the data has been clustered and the dendogram formed. 
\fixme{citation} By 
looking the dendogram and using different metrics the number of 
clustrs can be chosen.


\subsection{Evaluating of clustering results}
Calinski \& Harabazt criterion can be used as an evaluation 
criterion for deciding the number clusters \fixme{citation}. 

\subsubsection{Manually annotated validation set}
% Gold standard set. Actually a gold standard set would be a set
% of all data sets with abstract excluding sets that don't have it.
We will create a manually annotated validation set for calculating
precision, recall and metrics derived from those. The validation
set consists of $500$ publications from three different fields of
science, two more similar sub fields of computer science, 
information systems and artifical intelligence, and one more
distant from those, clinical neurology.

Publications are inspected by title, abstract, keywords, journal
and publisher assinged disciplines of the journal. Publications
with critically missing data, unclear discipline assignment and
heavily applied publications were excluded from validation set.
Goal was to achieve evaluation measurements based on a quite 
clearly separated set of publications. More vaguely classificable
publications were included for comparison. For manually curated 
validation set with discipline assingment and justifications for
possible exclusion see appendix (Insert reference!).

When annotating publications, deciding if a publication belonged to
a discipline or not felt often quite difficult. Often the 
separation between disciplines felt quite arbitrary. For example
an article describing using wavelet transformation for coding noisy
images was decided to belong to CS information systems whereas an
article describing wavelet based corner detection using SVD was
decided to belong to CS artifical intelligence.
For CS artifical intelligence we mostly selected publications which
mentioned some dimensionalty reduction or machine learning related
term or concept.
CS information systems ended up being quite like some ``others'' or
``the rest'' dump class. It would have publications such as
``A reference model for conceptualising the convergence of 
telecommunications and datacommunications service platforms'',
``Developing a distributed meeting service to support mobile 
meeting participants'',
``On voice quality of IP voice over GPRS'',
