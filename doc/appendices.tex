\chapter{Manually annotated data set}
\label{chapter:first-appendix}
Title, resulting classification and taken decision of manually 
annotated publications.

\pgfplotstabletypeset[
    col sep=semicolon,
    verb string type,
    begin table=\begin{longtable},
    end table=\end{longtable},
    columns={Title, My class name, Clarification 3},
    columns/Title/.style={column name=Title, column type={|p{60mm}}},
    columns/My class name/.style={column name=Discipline, column type={|p{36mm}}},
    columns/Clarification 3/.style={column name=Clarification, column type={|p{28mm}|}},
    every even row/.style={before row={\rowcolor[gray]{0.9}}},
    every head row/.style={before row=\hline,after row=\hline},
    every head row/.append style={after row=\endhead},    
    every last row/.style={after row=\hline},
    ]{../data/baseline/groundtruth_labels_CS-AI-IS_CN_for_appendix.csv}

 
\chapter{Top terms}
\label{chapter:second-appendix}
\pgfplotstabletypeset[
    col sep=colon,
    verb string type,
    begin table=\begin{longtable},
    end table=\end{longtable},
%     label=\label{table:topterms\_hier}
    columns/cluster/.style={column name={Cluster}, column type={|l}},
    columns/top terms/.style={column name={Top terms}, column type={|p{115mm}|}},
    every even row/.style={before row={\rowcolor[gray]{0.9}}},
    every head row/.style={before row=\hline,after row=\hline},
    every head row/.append style={after row=\endhead},
    every last row/.style={after row=\hline}
]{../data/processed/topterms/12000-188-800-hierarchical-topterms.csv}



\chapter{Articles}
\label{chapter:appendix-articles}
Titles and existing WoS subject categories of five random 
publications from ten random clusters of total $188$ clusters 
created by Ward's method.

% JPL: data from: data/processed/12000-188-800-hierarchical-results.txt

\newcolumntype{g}{>{\columncolor{gray!20}}p{5.7cm}}
% \begin{longtable}{|p{0.9cm}|p{6.5cm}|p{5.0cm}|}
\begin{longtable}{|p{0.9cm}|g|g|}
\hline % The line on top of the table
\rowcolor{white}
  \textbf{Clust} & \textbf{Title} & \textbf{WoS category} \\
\hline 
\rowcolor{white}
  \multirow{ 5}{*}{\textbf{9}} & Size of environmental grain and resource matching & ECOLOGY \\
& A note on Matthias Sutter & POLITICAL SCIENCE  \\
\rowcolor{white}
  & National outdoor recreation demand and supply in Finland: an assessment project & FORESTRY  \\
& Industrial ecology of the paper industry &   \\
\rowcolor{white}
  & Sex differences in quality of life among allogeneic BMT recipients & PSYCHOLOGY; SOCIAL SCIENCES BIOMEDICAL  \\
\hline

\hline 
\multirow{ 5}{*}{\textbf{19}} & Neuron weight dynamics in the SOM and Self-Organized Criticality & COMPUTER SCIENCE ARTIFICIAL INTELLIGENCE; ENGINEERING ELECTRICAL ELECTRONIC \\
\rowcolor{white}
  & Indexing text with approximate q-grams & COMPUTER SCIENCE THEORY METHODS \\
& XML based text TV & COMPUTER SCIENCE INFORMATION SYSTEMS; ENGINEERING ELECTRICAL ELECTRONIC; TELECOMMUNICATIONS \\
\rowcolor{white}
  & Musical networks: Parallel distributed perception and performance. & PSYCHOLOGY EXPERIMENTAL; MUSIC \\
& PARNEU: general-purpose partial tree computer & COMPUTER SCIENCE HARDWARE ARCHITECTURE; COMPUTER SCIENCE THEORY METHODS; ENGINEERING ELECTRICAL ELECTRONIC \\
\hline

\hline 
\rowcolor{white}
  \multirow{ 5}{*}{\textbf{31}} & Citalopram controls phobic symptoms in patients with panic disorder: randomized controlled trial & NEUROSCIENCES PSYCHIATRY PSYCHIATRY \\
& Long-term treatment of psoriasis with calcipotriol scalp solution and cream & DERMATOLOGY VENEREAL DISEASES \\
\rowcolor{white}
  & Docetaxel, a promising novel chemotherapeutic agent in advanced breast cancer & ONCOLOGY \\
& Silica xerogel as an implantable carrier for controlled drug delivery - evaluation of drug distribution and tissue effects after implantation & ENGINEERING BIOMEDICAL; MATERIALS SCIENCE BIOMATERIALS \\
\rowcolor{white}
  & A double-blind, randomized study to compare the efficacy and safety of terbinafine (Lamisil (R)) with fluconazole (Diflucan (R)) in the treatment of onychomycosis & DERMATOLOGY VENEREAL DISEASES \\
\hline

\hline
\multirow{ 5}{*}{\textbf{44}} & Genetics of disease - Away from the beaten track - Editorial overview & BIOCHEMISTRY MOLECULAR BIOLOGY; BIOTECHNOLOGY APPLIED MICROBIOLOGY; CELL BIOLOGY; GENETICS HEREDITY \\
\rowcolor{white}
  & Different sensitivities of human and rat rho(1) GABA receptors to extracellular pH & NEUROSCIENCES; PHARMACOLOGY PHARMACY \\
& Life cycle profit calculations for lubrication & MATERIALS SCIENCE PAPER WOOD \\
\rowcolor{white}
  & Wheat and rye brans: What is the difference? & PLANT SCIENCES; FOOD SCIENCE TECHNOLOGY \\
& Salbutamol via Easyhaler (R) produces equivalent bronchodilation to terbutaline via Turbuhaler (R) following inhalation of a single dose of equipotent beta(2)-sympathomimetic & PHARMACOLOGY PHARMACY \\
\hline

\hline
\rowcolor{white}
  \multirow{ 5}{*}{\textbf{55}} & Long-term effects of vitamin E, vitamin C, and combined supplementation on urinary 7-hydro-8-oxo-2 '-deoxyguanosine, serum cholesterol oxidation products, and oxidation resistance of lipids in nondepleted men & HEMATOLOGY; PERIPHERAL VASCULAR DISEASE \\
& Factors affecting broadband ultrasound attenuation results of the calcaneus using a gel-coupled quantitative ultrasound scanning system & ENDOCRINOLOGY METABOLISM \\
\rowcolor{white}
  & Controlled trial of alpha-tocopherol and beta-carotene supplements on stroke incidence and mortality in male smokers & HEMATOLOGY; PERIPHERAL VASCULAR DISEASE \\
& Quercetin intake and the incidence of cerebrovascular disease & NUTRITION DIETETICS \\
\rowcolor{white}
  & Regional cerebral blood flow during exposure to food in obese binge eating women & CLINICAL NEUROLOGY; NEUROIMAGING; PSYCHIATRY \\
\hline

\hline
\multirow{ 5}{*}{\textbf{61}} & An engineering model for heating energy and emission assessment - The case of North Karelia, Finland & ENERGY FUELS; ENGINEERING CHEMICAL \\
\rowcolor{white}
  & Lattice sites of diffused gold and platinum in epitaxial ZnSe layers &  \\
& Pevotella pallens has a unique phospholipid analogue profile. & DENTISTRY ORAL SURGERY MEDICINE \\
\rowcolor{white}
  & Heat transfer enhancement at solid-liquid and solid-gas interfaces by near-surface coolant agitation & ENGINEERING MANUFACTURING; ENGINEERING ELECTRICAL ELECTRONIC; MATERIALS SCIENCE MULTIDISCIPLINARY \\
& Peaked density profiles: evidence of inward transport in the W7-AS stellarator & PHYSICS FLUIDS PLASMAS; PHYSICS NUCLEAR \\
\hline

\hline
\rowcolor{white}
  \multirow{ 5}{*}{\textbf{93}} & Non-stationary Alfven resonator: vertical profiles of wave characteristics & GEOCHEMISTRY GEOPHYSICS; METEOROLOGY ATMOSPHERIC SCIENCES \\
& Unified description of nondiffracting X and Y waves & PHYSICS FLUIDS PLASMAS; PHYSICS MATHEMATICAL \\
\rowcolor{white}
  & Generation of artificial magnetic pulsations in the Pc1 frequency range by periodic heating of the Earth's ionosphere: indications of ionospheric Alfven resonator effects &  \\
& 2D and 3D models for the cross-talk modeling in acoustic devices: A fast-MoM approach & ACOUSTICS; INSTRUMENTS INSTRUMENTATION; PHYSICS APPLIED \\
\rowcolor{white}
  & Non-stationary Alfven resonator: new results on Pc1 pearls and IPDP events &  \\
\hline

\hline
\multirow{ 5}{*}{\textbf{131}} & Search for schizophrenia genes in Finnish families reveal a locus on Chromosome 1q & GENETICS HEREDITY \\
\rowcolor{white}
  & FMR1 CGG expansion to full mutation: What is the lower limit in premutation females? & GENETICS HEREDITY \\
& High-throughput gene copy number analysis in 4700 tumors: FISH analysis on tissue microarrays identifies multiple tumor types with amplification of the MB-174 gene, a novel amplified gene originally found in breast cancer. & GENETICS HEREDITY \\
\rowcolor{white}
  & DTDST is expressed in developing fetal cartilage but also in a wide variety of other tissues and cell types. & GENETICS HEREDITY \\
& Relative fitness of women with the mitochondrial DNA mutation 3243AG. & GENETICS HEREDITY \\
\hline

\hline
\rowcolor{white}
  \multirow{ 5}{*}{\textbf{143}} & MEG and MCG in a clinical environment: BioMag Laboratory, Helsinki & BIOPHYSICS; ENGINEERING BIOMEDICAL; INSTRUMENTS INSTRUMENTATION \\
& Spatial aspects in magnetocardiographic time domain and spectrotemporal post-infarction arrhythmia risk assessment & BIOPHYSICS; ENGINEERING BIOMEDICAL; INSTRUMENTS INSTRUMENTATION \\
\rowcolor{white}
  & High performance magnetically shielded room for clinical measurements & BIOPHYSICS; ENGINEERING BIOMEDICAL; INSTRUMENTS INSTRUMENTATION \\
& Information theory predicts the magnitude of an MEG response related to the operation of the auditory sensory memory & BIOPHYSICS; ENGINEERING BIOMEDICAL; INSTRUMENTS INSTRUMENTATION \\
\rowcolor{white}
  & Global optimization in the localization of brain activity & BIOPHYSICS; ENGINEERING BIOMEDICAL; INSTRUMENTS INSTRUMENTATION \\
\hline

\hline
\multirow{ 5}{*}{\textbf{144}} & Cotinine and nicotine inhibit each other's calcium responses in bovine chromaffin cells & PHARMACOLOGY PHARMACY; TOXICOLOGY \\
\rowcolor{white}
  & Comparison of the effects of epibatidine and nicotine on the output of dopamine in the dorsal and ventral striatum of freely-moving rats & PHARMACOLOGY PHARMACY \\
& Effect of acute nicotine administration on striatal dopamine output and metabolism in rats kept at different ambient temperatures & BIOCHEMISTRY MOLECULAR BIOLOGY; PHARMACOLOGY PHARMACY \\
\rowcolor{white}
  & The effects of acute nicotine on the body temperature and striatal dopamine metabolism of mice during chronic nicotine infusion & NEUROSCIENCES \\
& Nicotine-evoked exocytosis from bovine chromaffin cells is independent of phospholipase D activation & NEUROSCIENCES \\
\hline
\hline
\end{longtable}

