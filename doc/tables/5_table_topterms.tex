\begin{table}
\begin{tabular}{|p{2cm}|p{10.5cm}|} 
% Alignment of sells: l=left, c=center, r=right. 
% If you want wrapping lines, use p{width} exact cell widths.
% If you want vertical lines between columns, write | above between the letters
% Horizontal lines are generated with the \hline command:
\hline % The line on top of the table
\textbf{ } & \textbf{Top terms} \\ 
\hline 
\textbf{Cluster 0} & system internet management user project software architecture support decision business channel rule concept quality exercise  \\ 
\hline
\hline 
\textbf{Cluster 1} & lenges professional care intellectual disability people service special syndrome life child social support diagnosis pathology  \\ 
\hline
\hline 
\textbf{Cluster 2} & epilepsy stroke child seizure headache neuronal p rat sleep nerve risk treatment cell alcohol activity  \\ 
\hline
\hline 
\textbf{Cluster 3} & component filter signal neural linear bayesian ica feature value input simulation independent fuzzy vector output \\ 
\hline
\hline 
\textbf{Cluster 4} & dementia vascular ad cognitive stroke alzheimer diagnosis lesion vad risk pd brain impairment parkinson aneurysm \\ 
\hline
\hline 
\textbf{Cluster 5} & service development mobile switch agent system solution traffic environment protocol ip technology wireless game implementation \\ 
\hline
\hline 
\textbf{Cluster 6} & map image som self-organizing query logic document tree k expressive compression database retrieval rule cluster \\ 
\hline
\end{tabular} % for really simple tables, you can just use tabular
% You can place the caption either below (like here) or above the table
\caption{Top terms per cluster for manually annotated samples}
% Place the label just after the caption to make the link work
\label{table:topterms}
\end{table} % table makes a floating object with a title
