\chapter{Methods}
\label{chapter:methods}
In this chapter we present the methods used in the clustering. We 
follow the logical order the methods are applied on data.

\section{SVD}
\label{sec:svd}
There are multiple ways to use input metadata of the publications; 
put all text to one bin and use text analysis methods to that. 
Alternatively we could assume different distribution for each 
metadata field and treat them separately.

\section{Agglomerative clustering}
\label{sec:agglomerativeclustering}
There are also multiple methods to use for the problem. LDA is one.

\subsection{Distance metric}
Data is high dimensional so choosing distance measure is 
important. The higher the number of dimensions the more similar 
distance each observation is from every other observation. This 
known as \emph{curse of dimensionality} \fixme{citation}.
Possible distance measures are:
- cosine angle (uncentered Pearson correlation)\fixme{look: 
\url{https://www.researchgate.net/post/What_is_the_best_distance_
measure_for_high_dimensional_data/4}}
-euclidean
-mahalnobis

\subsection{Linkage methods}
In this work we use agglomerative hierarchical clustering with 
Ward's distance metric.\fixme{citation}

\subsubsection{Single linkage}
\subsubsection{Complete linkage}
\subsubsection{Average linkage}
\subsubsection{Ward's method}

\subsection{Complexity}
Time and space requirements of used agglomerative clustering method
are $O(N^2)$ for both.\fixme{citation}


