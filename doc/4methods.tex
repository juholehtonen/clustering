\chapter{Data and Methods}
\label{chapter:methods}
In this chapter we present the data and methods used in the 
clustering. We follow the logical order the methods are applied on 
the data.

\section{Publication metadata}
The data consits of X records of Web of Science publication 
metadata from year Y.

\section{Analyzing textual data}
Our data here consists of short metadata records describing the 
publications. Basically the descriptions are natural English 
language. When analyzing this kind of textual data the often used
methods involve some kind of counting. We count for example which
are the most used words, which words appear together and so on.

We can count n-grams.
\subsection{Preprocessing}
\subsection{Lemmatizing}

\subsection{Different distributions of different metadata fields}
\subsubsection{Keyword distribution}
\subsubsection{Citation distribution}
- Normalization of citation weight between different disciplines.
  Different fields of science have different citation practices 
eg. regarding the number of references per publication. 
\cite{waltman_new_2012}


\section{SVD}
\label{sec:svd}
There are multiple ways to use input metadata of the publications; 
put all text in one bin and use text analysis methods to that. 
Alternatively we could assume different distribution for each 
metadata field \fixme{describe metadata before this} and treat them
separately.

\section{Agglomerative clustering}
\label{sec:agglomerativeclustering}
There are also multiple methods to use for the problem. LDA is one.

\subsection{Distance metric}
Data is high dimensional so choosing distance measure is 
important. The higher the number of dimensions the more similar 
distance each observation is from every other observation. This is
known as \emph{curse of dimensionality} \fixme{citation}.
Possible distance measures are:
- cosine angle (uncentered Pearson correlation)\fixme{look: 
\url{https://www.researchgate.net/post/What_is_the_best_distance_
measure_for_high_dimensional_data/4}}
-euclidean
-mahalnobis

\subsection{Linkage methods}
In this work we use agglomerative hierarchical clustering with 
Ward's distance metric.\fixme{citation}

\subsubsection{Single linkage}
\subsubsection{Complete linkage}
\subsubsection{Average linkage}
\subsubsection{Ward's method}

\subsection{Complexity}
Time and space requirements of used agglomerative clustering method
are $O(N^2)$ for both.\fixme{citation}


