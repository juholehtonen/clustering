\chapter{Discussion}
\label{chapter:discussion}

The usefulness of my results lies in that they show how difficult
the task at hand is and what aspects there are to consider when
approaching the problem.
% Alkuasetelma
The problem of defining how the science should be classified into
different fields of science is quite difficult and ambiguous.
% Datasta
Although the data was incomplete we chose to use it as such with out 
pruning the bad samples. This was 
Filtering out the bad samples could have improved the clustering 
results.
% Piirreirroituksesta
We tested some hand picked values of the feature extraction 
parameters, minimum and maximum document frequency threshold for 
terms. More systematic exploration of these and other 
pre-processing related parameters could give robust results.
- \fixme{``Suositus...'' ``Nämä parametriarvot...''}
% Ulottuvuuksien vähentämisestä
Selected number of reduced dimensions seemd to retain enough
information while also allowing to run clustrings fast enough.
% Model selection
The model selected, agglomerative Ward's clustering, is one of the
basic clustering methods. It is known to work well with ...
while failing when ....
Some more elaborated method could improve the results. 
- \fixme{Vaihtoehtojen kokeiluiden havainnot. Linkitys 
kirjallisuuteen.}
% Itse klusteroinnista

- Compared to current classification of articles into fields of 
science based on WoS classification...


% Toteutuksen heikkoudet ja vahvuudet

% Vertailu aiempaa suomalaiseen vastaavaan ``Suominen... et al.''
