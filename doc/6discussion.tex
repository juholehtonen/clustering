\chapter{Discussion}
\label{chapter:discussion}

The usefulness of my results lies in that they show how difficult
the task at hand is and what aspects there are to consider when
approaching the problem.
% Alkuasetelma
The problem of defining how the science should be classified into
different fields of science is quite difficult and ambiguous.

% Datasta
Although the data was incomplete we chose to use it as such without 
pruning the possible defective samples. This was due to the 
amount of data, manually checking over 10000 samples wasn't an 
option and general rules to filter the data didn't occur in the
beginning of this work. Some minimum length for the publication 
data could leave out clearly inadequate samples and improve the 
clustering results. General outlier detection techniques 
\cite{hodge_survey_2004} could prove 
problematic here when there obviously will be single publications 
in some expected disciplines. Even including more data there
probably could be disciplines with very few publications.

% Piirreirroituksesta
We tested some hand picked values of the feature extraction 
parameters, minimum and maximum document frequency threshold for 
terms. More systematic exploration of these and other 
pre-processing related parameters could give robust results.
Tokenization during preprocessing left one character tokens (words)
in data which wasn't desired.
- \fixme{``Suositus...'' ``Nämä parametriarvot...''}

% Ulottuvuuksien vähentämisestä
Selected number of reduced dimensions seemd to retain enough
information while also allowing to run clustrings fast enough.
The internal validation metrics might still have had trouble to 
attain higher values with respect their scale with the fact that 
$800$ dimensional data with just $455$ samples still probably is 
too high dimensional \cite{aggarwal_surprising_2001}. 
Reducing dimensionality more could have helped the internal 
validation metrics to perform better, that is to find ... (vai 
itse klusterointi olisi toiminut ``paremmin'' (vähemmällä informaatiolla tosin)?)
Transforming data somewhere to $[50-150]$ components could be more
justified range \cite{dolnicar_review_2002}. 

% Model selection
The model selected, agglomerative Ward's clustering, is one of the
basic clustering methods. It is known to work well with compact,
even-sized clusters.
% while failing ...

% Choosing the number of clusters
Our manually annotated data set for choosing the number of clusters
was propably too small ($455$) compared to number of components 
($800$). Some more elaborated method could improve the results. 
We also have to remember that Calinski-Harabasz index might suffer
even from moderate noise \cite{liu_understanding_2010}.
There are other internal clustering validation metrics that might 
work better with this kind of data \cite{liu_understanding_2010}.
- \fixme{Vaihtoehtojen kokeiluiden havainnot. Linkitys 
kirjallisuuteen.}

% Itse klusteroinnista
Using the actual hierarchical structure of resulting clustering 
was unfortunately not utilized here. It could have helped in 
deciding the number of clusters \cite{kimes_statistical_2017}.
- Compared to current classification of articles into fields of 
science based on WoS classification...


% Toteutuksen heikkoudet ja vahvuudet

% Vertailu aiempaa suomalaiseen vastaavaan ``Suominen... et al.''


\section*{Conclusions}
\label{sec:conclusions}
In future work 2-grams or higher order n-grams could improve the 
results.
% Two to four pages might be a good limit. 


\section*{Future work}
\label{sec:future}
In future work 2-grams or higher order n-grams could improve the 
results.
